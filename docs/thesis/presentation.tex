\documentclass{beamer}
\usepackage{indentfirst}
\usepackage[italian]{babel}
\usepackage{newlfont}
\usepackage{physics}
\usepackage{amsmath}
\usepackage{amssymb}
\usepackage{amsthm}
\usepackage{xcolor}
\usepackage{listings}
\input{./solidity-highlighting.tex}
\usepackage{tikz}
\usepackage{xspace}
\usepackage{adjustbox}

\usepackage[all]{xy}
\usepackage{array}
\usepackage{xcolor}

\newcommand{\cl}[1]{{\cal #1}}         
\newcommand{\w}[1]{{\it #1}}

\definecolor{primary}{RGB}{61,50,204} 

\makeatletter
\newcommand\listofframes{\@starttoc{lbf}}
\makeatother

\addtobeamertemplate{frametitle}{}{%
  \addcontentsline{lbf}{section}{\protect\makebox[2em][l]{
    \protect\usebeamercolor[fg]{structure}}
  \insertframetitle\par}
}

\usetheme{Boadilla}
\usecolortheme[RGB={61,50,204}]{structure}

\title{Calcolo del Cost Model Esatto \\di un Programma in Solidity}
\author{Mattia Guazzaloca}
\institute{Alma Mater Studiorum - Università di Bologna}
\date{13 Ottobre 2021}

\begin{document}
\abovedisplayskip=0pt
\abovedisplayshortskip=0pt
\belowdisplayskip=0pt
\belowdisplayshortskip=0pt
\begin{frame}
\titlepage
\end{frame}

\begin{frame}
    \frametitle{Sommario}
    \tableofcontents
\end{frame}

\section{Background}
\begin{frame}
    \frametitle{Background}
    \begin{block}{Ethereum}
    Ethereum \`{e} una \textbf{piattaforma} open-source basata su blockchain per lo sviluppo di applicazioni distribuite denominate \textbf{smart contracts}.
    \end{block}
    \begin{block}{L'Ether e il Gas}
    L'\textbf{Ether} (ETH) \`{e} la criptovaluta di Ethereum. Viene usata specialmente per l'acquisto del \textbf{gas}, l'unit\`{a} di misura dello sforzo computazionale richesto per l'esecuzione di uno smart contract.
    \end{block}
    \begin{block}{Solidity}
    \textbf{Solidity} \`{e} il linguaggio di alto livello per lo sviluppo di smart contract pi\`{u} utilizzato. \`{E} object-oriented e la sua sintassi \`{e} largamente ispirata a quella del linguaggio JavaScript.
    \end{block}
\end{frame}

\section{CerCo}
\begin{frame}
    \frametitle{CerCo}
    CerCo \`{e} un progetto europeo di ricerca nato con lo scopo di sviluppare e provare formalmente un \textbf{cost annotating compiler}.
    \begin{block}{Cost Annotating Compiler}
    Un \textit{cost annotating compiler} $C$ \`{e} un compilatore che preso in input un programma $P$ produce in output un programma $An(P)$ "annotato", funzionalmente equivalente a $P$, ma capace di tracciare il costo della sua esecuzione.\\
    \end{block}
    Le informazioni che il compilatore $C$ inserisce nel programma annotato $An(P)$ per tracciare l'esecuzione si chiamano \textbf{annotazioni di costo}.
\end{frame}
% \begin{frame}{Correttezza delle Cost Annotation}
% Un'annotazione di costo \`{e} corretta se e solamente se soddisfa sia la propriet\`{a} di \textit{soundness} che quella di \textit{preciseness}.
% \begin{block}{Soundness}
% Una cost annotation \`{e} \textit{sound} se il suo valore costituisce un upper bound al costo di esecuzione reale
% \end{block}
% \begin{block}{Preciseness}
% Una cost annotation \`{e} \textit{precise} quando la differenza fra il costo reale dell'esecuzione e quello espresso dall'annotazione \`{e} limitata dal valore di una costante che dipende solo dal programma
% \end{block}
% \end{frame}
\begin{frame}
    \frametitle{Il Metodo del Labelling}
    \begin{center}
    \begin{tabular}{m{10cm}}
    $$
    \xymatrix{
    % % Row 2
      L_1 &
      \ar[l]^{\cl{I}}
      L_{1,\ell} 
      \ar@/^/[d]^{er_1} 
      \ar[r]^{\cl{C}_1}
    %
    & L_{2,\ell} 
      \ar[d]^{er_2}  
    %
    & \ldots \hspace{0.3cm}\ar[r]^{\cl{C}_k} 
    %
    & L_{k+1,\ell} 
      \ar[d]^{er_{k+1}}  \\
    %
    % Row 3
    &
      L_1                                  
      \ar@/^/[u]^{\cl{L}} 
      \ar[r]^{\cl{C}_1}
    & L_2   
    & \ldots\hspace{0.3cm}
      \ar[r]^{\cl{C}_{k}}
    & L_{k+1}
    }
    $$
    \end{tabular}
    \end{center}
    \begin{block}{Labelling}
    Un \textit{labelling} $L$ di un linguaggio sorgente $L_i$ \`{e} una funzione tale che $er_{L_i}{\circ}L$ \`{e} la funzione identit\`{a}
    \end{block}
    \begin{block}{Instrumentazione}
    Una \textit{instrumentazione} $I$ di un linguaggio etichettato $L_{1,\ell}$ \`{e} una funzione che rimpiazza le label di $L_{1,\ell}$ con, per esempio, incrementi di una variabile di costo
    \end{block}
\end{frame}

\section{Implementazione}
\begin{frame}
    \frametitle{Implementazione}
    \begin{block}{Scopo della Tesi}
    Realizzare un \textbf{cost annotating compiler} per il linguaggio Solidity applicando l'approccio sviluppato da CerCo.
    \end{block}
    \begin{block}{Risultato}
    La complessit\`{a} di Solidity e le peculiarit\`{a} degli smart contract hanno generato problemi nuovi rispetto a CerCo e sono causa delle limitazioni dell'attuale implementazione.
    \end{block}
\end{frame}
\begin{frame}[fragile]
    \frametitle{Labelling}
    \begin{columns}
        \begin{column}{0.4\textwidth}
            \begin{itemize}
                \item L'inizio di \textbf{ciascun blocco} \`{e} etichettato con una label
                \item Tutte le sequenze di istruzioni precedute da un qualche costrutto per il \textbf{controllo di flusso} sono etichettate con una label posta subito dopo la fine del costrutto
            \end{itemize}
        \end{column}
        \begin{column}{0.05\textwidth}
        \end{column}
        \begin{column}{0.55\textwidth}
            \begin{adjustbox}{max height=3.5cm,keepaspectratio}
            \begin{lstlisting}[language=Solidity,frame=trlb,linewidth=9.8cm]
contract Fibonacci {
  function fibonacci(uint256 n) public pure returns (uint256 b) {
    /* __cost0 */
    if (n == 0) {
      /* __cost1 */
      return 0;
    }
    /* __cost2 */
    uint256 a = 1;
    b = 1;
    for (uint256 i = 2; i < n; i++) {
      /* __cost3 */
      uint256 c = a + b;
      a = b;
      b = c;
    }
    /* __cost4 */
    return b;
  }
}
            \end{lstlisting}
            \end{adjustbox}
            \end{column}
    \end{columns}
\end{frame}
\begin{frame}
    \frametitle{Calcolo dei Costi}
    % Definizione super ridotta della funzione di costo del gas e come si è (non) risolto il problema
    \begin{block}{Funzione di Costo del Gas}
    La funzione di costo del gas $C(\boldsymbol{\sigma}, \boldsymbol{\mu}, w)$ associa all'istruzione $w$ il consumo di gas dato lo stato della blockchain $\boldsymbol{\sigma}$ e quello della EVM $\boldsymbol{\mu}$:
    \begin{equation*}
    \resizebox{.8\textwidth}{!}
    {$
    C(\boldsymbol{\sigma}, \boldsymbol{\mu}, w) \equiv C_{\mathrm{mem}}(\boldsymbol{\mu}'_{\mathrm{i}})-C_{\mathrm{mem}}(\boldsymbol{\mu}_{\mathrm{i}}) + 
    \begin{cases}
    C_\text{\tiny SSTORE}(\boldsymbol{\sigma}, \boldsymbol{\mu}) & \text{if} \quad w = \text{\small SSTORE} \\
    G_{\mathrm{verylow}} + G_{\mathrm{copy}}\times\lceil\boldsymbol{\mu}_{\mathbf{s}}[2] \div 32\rceil & \text{if} \quad w \in W_{\mathrm{copy}} \\
    C_\text{\tiny CALL}(\boldsymbol{\sigma}, \boldsymbol{\mu}) & \text{if} \quad w \in W_{\mathrm{call}} \\
    C_\text{\tiny SELFDESTRUCT}(\boldsymbol{\sigma}, \boldsymbol{\mu}) & \text{if} \quad w = \text{\small SELFDESTRUCT} \\
    G_{\mathrm{create}} & \text{if} \quad w = \text{\small CREATE}\\
    G_{\mathrm{zero}} & \text{if} \quad w \in W_{\mathrm{zero}}\\
    G_{\mathrm{base}} & \text{if} \quad w \in W_{\mathrm{base}}
    \end{cases}
    $}
    \end{equation*}
    dove:
    \begin{equation*}
    \resizebox{.4\textwidth}{!}
    {$
    C_{\mathrm{mem}}(a) \equiv G_{\mathrm{memory}} \cdot a + \left\lfloor \dfrac{a^2}{512} \right\rfloor
    $}
    \end{equation*}
    \end{block}
    Soluzione adottata:
    \begin{itemize}
        \item Simulazione \textbf{parziale} dell'esecuzione
        \item Assegnamento diretto dei consumi per le istruzioni con costo costante
        \item Rapprensetazione di $C(\boldsymbol{\sigma}, \boldsymbol{\mu}, w)$ per le istruzioni pi\`{u} complesse
    \end{itemize}
\end{frame}
\begin{frame}
    \frametitle{Attribuzione dei Costi}
    \begin{enumerate}
        \item
        Si costruisce il \textbf{Control Flow Graph} del sorgente assembly
        \begin{block}{Control Flow Graph}
        Dato un qualche programma $P$, si definisce \textit{Control Flow Graph} il grafo orientanto $G = (N,E)$ dove ciascun nodo $n \in N$ \`{e} un blocco del programma $P$ e ogni arco $(n_i,n_j) \in E$ rappresenta il passaggio del flusso di esecuzione dal nodo $n_i$ al nodo $n_j$.
        \end{block}
        \item
        Si visita in profondit\`{a} il CFG mantenendo uno stack $\mu$ delle label raggiunte. Durante la visita:
        \begin{itemize}
            \item i nodi \textbf{etichetta}, non precedentemente visitati, vengono aggiunti a $\mu$
            \item il consumo di gas dei nodi \textbf{istruzione} viene sommato all'etichetta che si trova in cima a $\mu$
        \end{itemize}
    \end{enumerate}
\end{frame}
\begin{frame}
    \frametitle{Costo dell'Inizializzazione}
    Ogni smart contract necessita di codice aggiuntivo per:
    \begin{itemize}
        \item Il deploy, i.e. il caricamento del contratto sulla blockchain
        \item La corretta invocazione dei metodi parte dell'API pubblica del contratto
    \end{itemize}
    Il codice generato non \`{e} coperto da label e genera quindi annotazioni di costo \textit{unsound}.
    \begin{block}{\texttt{startupCost}}
    L'algoritmo per l'attribuzione dei costi definisce una variabile ausiliaria \texttt{startupCost} che tiene traccia del costo dell'inizializzazione. Il suo valore viene assegnato alla prima label incontrata durante la visita.
    \end{block}
\end{frame}
\begin{frame}[fragile]
    \frametitle{Instrumentazione}
    % Due "side":
    % - sx: descrizione per punti dell'instrumentazione [var di costo, machinery, gestione funzioni public]
    \begin{columns}
        \begin{column}{0.4\textwidth}
            \begin{enumerate}
                \item Aggiunta del codice per definire ed accedere alla \textbf{variabile di costo}
                \item Sostituzione delle label con i corrispettivi \textbf{incrementi} della variabile di costo
            \end{enumerate}
        \end{column}
        \begin{column}{0.05\textwidth}
        \end{column}
        \begin{column}{0.55\textwidth}
            \begin{adjustbox}{max height=4cm,margin=0pt,keepaspectratio}
            \begin{lstlisting}[language=Solidity,frame=trlb,linewidth=12.3cm]
contract SimpleStorageInstrumented {
    uint256 storedData;

    // NOTE: was a view function
    function get() internal returns (uint256) {
        /* __cost1 */
        __costAcc = __costAcc + 874; /* + memaccess(128, add(not(127),abi_encode_uint256(sload(0)))) */
        return storedData;
    }

    function get_external() external returns (uint256) {
        /* __cost1.startup */
        __costAcc = __costAcc + 155;
        return get();
    }
    
    // Instrumentation Machinery
    uint256 private __costAcc = 0;
    function __SimpleStorage_getCost() public view returns(uint256) { 
        return __costAcc; 
    }
    function __SimpleStorage_resetCost() public {
        __costAcc = 0;
    }
}
            \end{lstlisting}
            \end{adjustbox}
            \end{column}
    \end{columns}
\end{frame}

\section{Conclusioni}
\begin{frame}
    \frametitle{Conclusioni}
    Stimare in maniera chiara e precisa i consumi di gas degli smart contract \`{e} un compito non triviale, specialmente quando questi sono scritti in un linguaggio di alto livello come Solidity. L'implementazione presentata risolve in parte il problema grazie al lavoro svolto dal progetto CerCo.\\
    \vspace{1em}
    Attuali limitazioni:
    \begin{enumerate}
        \item L'instrumentazione \`{e} \textbf{incompleta}: manca ad esempio il supporto per le \texttt{library}
        \item Il calcolo dei consumi di gas rappresenta solo un'indicazione allo sviluppatore per via di quelle istruzioni il cui costo dipende dallo stato della EVM
    \end{enumerate}
    \vspace{1em}
    Inoltre, principalmente a causa del punto 2, il testing dell'implementazione \`{e} avvenuto solo in maniera parziale.
\end{frame}

\begin{frame}
    \frametitle{Metriche sul Lavoro Svolto}
    In totale lo sviluppo dell'elaborato ha richiesto all'incirca \textbf{5 mesi}, da maggio a settembre compresi, per quanto le ore effettive di lavoro siano chiaramente in numero inferiore, ma difficilmente quantificabili.\\
    \vspace{1em}
    La scrittura del codice ha riguardato due aree distinte:
    \begin{itemize}
        \item Il \textbf{compilatore di Solidity} (solc): modifiche minime al sorgente in C++ per un totale di circa \textbf{100-200 LoC}.
        \item \textbf{Cerco2}: tool esterno al compilatore principalmente legato alla fase di \textbf{calcolo e attribuzione dei costi}. Questo tool \`{e} stato scritto in JavaScript ed ha una dimensione pari a circa \textbf{2000 LoC}.
    \end{itemize}
\end{frame}
\begin{frame}
    \centering
    \fontsize{24pt}{0pt}\selectfont
    Grazie!
\end{frame}
\end{document}